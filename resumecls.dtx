% \iffalse meta-comment
%
% resumecls.dtx
% Copyright 2012 by huxuan <i@huxuan.org>
%
% This work may be distributed and/or modified under the
% conditions of the LaTeX Project Public License, either version 1.3
% of this license or (at your option) any later version.
% The latest version of this license is in
%   http://www.latex-project.org/lppl.txt
% and version 1.3 or later is part of all distributions of LaTeX
% version 2005/12/01 or later.
%
% This work has the LPPL maintenance status `maintained'.
%
% The Current Maintainer of this work is huxuan <i@huxuan.org>.
%
% This work consists of the files resumecls.dtx and resumecls.ins
% and the derived file resumecls.cls.
%
% \fi

% \iffalse
%<cls>\NeedsTeXFormat{LaTeX2e}[2011/06/27]
%<cls>\ProvidesClass{resumecls}
%<cls>[2012/12/20 v0.1.4 Use bfseries for better English support]
%
%<*driver>
\documentclass{ltxdoc}
\usepackage[adobefonts]{ctex}

\usepackage{color}
\definecolor{dkgreen}{rgb}{0,0.6,0}
\definecolor{gray}{rgb}{0.5,0.5,0.5}
\definecolor{mauve}{rgb}{0.58,0,0.82}

\usepackage[xetex,unicode,colorlinks]{hyperref}

\usepackage{listings}
\lstset{
    backgroundcolor=\color{white},
    basicstyle=\zihao{5}\ttfamily,
    columns=flexible,
    breakatwhitespace=false,
    breaklines=true,
    captionpos=b,
    frame=single,
    numbers=left,
    numbersep=5pt,
    showspaces=false,
    showstringspaces=false,
    showtabs=false,
    stepnumber=1,
    rulecolor=\color{black},
    tabsize=2,
    texcl=true,
    title=\lstname,
    escapeinside={\%*}{*)},
    extendedchars=false,
    mathescape=true,
    xleftmargin=3em,
    xrightmargin=3em,
    numberstyle=\color{gray},
    keywordstyle=\color{blue},
    commentstyle=\color{dkgreen},
    stringstyle=\color{mauve},
    language={[LaTeX]TeX},
    morekeywords={setmainfont,setCJKmainfont},
}
\EnableCrossrefs
\CodelineIndex
\RecordChanges
%\OnlyDescription
\begin{document}
    \DocInput{\jobname.dtx}
\end{document}
%</driver>
% \fi
%
% \CheckSum{139}
%
% \CharacterTable
%  {Upper-case    \A\B\C\D\E\F\G\H\I\J\K\L\M\N\O\P\Q\R\S\T\U\V\W\X\Y\Z
%   Lower-case    \a\b\c\d\e\f\g\h\i\j\k\l\m\n\o\p\q\r\s\t\u\v\w\x\y\z
%   Digits        \0\1\2\3\4\5\6\7\8\9
%   Exclamation   \!     Double quote  \"     Hash (number) \#
%   Dollar        \$     Percent       \%     Ampersand     \&
%   Acute accent  \'     Left paren    \(     Right paren   \)
%   Asterisk      \*     Plus          \+     Comma         \,
%   Minus         \-     Point         \.     Solidus       \/
%   Colon         \:     Semicolon     \;     Less than     \<
%   Equals        \=     Greater than  \>     Question mark \?
%   Commercial at \@     Left bracket  \[     Backslash     \\
%   Right bracket \]     Circumflex    \^     Underscore    \_
%   Grave accent  \`     Left brace    \{     Vertical bar  \|
%   Right brace   \}     Tilde         \~}
%
% \changes{v0.1}{2012/12/19}{Initial version with dtx}
% \changes{v0.1.1}{2012/12/19}{Minor bug fix}
% \changes{v0.1.2}{2012/12/19}{Customization part in documentation}
% \changes{v0.1.3}{2012/12/19}{Add reference settings}
% \changes{v0.1.4}{2012/12/20}{Use bfseries for better English support}
%
% \GetFileInfo{\jobname.dtx}
%
% \DoNotIndex{\\,\begin,\colorbox,\CTEXoptions,\CurrentOptionn,\def}
% \DoNotIndex{\definecolor,\else,\end,\fancyfoot,\fancyhf,\fi,\footnotesize}
% \DoNotIndex{\footrulewidth,\headrulewidth,\heiti,\href,\hspace,\hypersetup}
% \DoNotIndex{\ifrcls@color,\ifrcls@en,\ifrcls@zh,\LoadClass,\maketitle}
% \DoNotIndex{\newcommand,\newif,\pagestyle,\parbox,\PassOptionsToClass}
% \DoNotIndex{\ProcessOptions,\refname,\relax,\renewcommand,\RequirePackage}
% \DoNotIndex{\textwidth,\today,\url,\zihao}
% \DoNotIndex{}
%
% \def\resumecls{\textsf{resumecls}\ }
%
% \title{The \resumecls package\thanks{This document
% corresponds to \resumecls~\fileversion,
% dated~\filedate.}}
% \author{huxuan \\ \texttt{i@huxuan.org}}
%
% \maketitle
%
% \begin{abstract}
% \resumecls is a latex cls to create a resume or cv more easily.
% Especially it supports Chinese as well as English at the same time.
% \end{abstract}
%
% \section{Introduction}
%
% There do exist many resume cls files already. But from my experience,
% most of them are too complicated or lack of flexibility.
%
% \resumecls make all contents consists of heading/entry based on tabularx.
% You can easily design the style and many recommended ones are also listed
% in example files.
%
% What's more, \resumecls already has ctex package imported which make it
% support Chinese natively. The example-zh.tex and example-en.tex is almost
% the same (in \LaTeX{} Code, not the content). You need no more work
% to make your resume both in English and Chinese.
%
% \section{Usage}
%
% \DescribeMacro{\heading}
% The heading part for each section.
%
% \DescribeMacro{\entry}
% All contents except for heading.
%
% \DescribeMacro{\name}
% Your Name.
%
% \DescribeMacro{\organization}
% Your unit, shcool or organization.
%
% \DescribeMacro{\address}
% Your address and zip code.
%
% \DescribeMacro{\mobile}
% Your phone number.
%
% \DescribeMacro{\mail}
% Your mail address.
%
% \DescribeMacro{\homepage}
% Your homepage.
%
% \DescribeMacro{\resumeurl}
% The url for the resume.
% It will locate at the right of footer.
% If you don't want it, just leave it blank.
%
% \StopEventually{\PrintChanges\PrintIndex}
%
% \section{Implementation}
%
% \subsection{Options}
%    \begin{macrocode}
\newif\ifrcls@zh\rcls@zhtrue
\newif\ifrcls@en\rcls@enfalse
\newif\ifrcls@color\rcls@colorfalse
\DeclareOption{zh}{\rcls@zhtrue\rcls@enfalse}
\DeclareOption{en}{\rcls@entrue\rcls@zhfalse}
\DeclareOption{color}{\rcls@colortrue}
%    \end{macrocode}
%
% \subsection{Import article class}
%    \begin{macrocode}
\DeclareOption*{\PassOptionsToClass{\CurrentOption}{article}}
\ProcessOptions\relax
\LoadClass[a4paper,12pt]{article}
%    \end{macrocode}
%
% \subsection{Import packages}
%    \begin{macrocode}
\RequirePackage[top=.5in,bottom=.5in,left=.5in,right=.5in]{geometry}
\RequirePackage[xetex,unicode]{hyperref}
\RequirePackage[noindent,adobefonts]{ctex}
\RequirePackage{tabularx}
\RequirePackage{color}
\RequirePackage{fancyhdr}
%    \end{macrocode}
%
% \subsection{Color Settings}
%
% \subsubsection{Background color for heading}
%    \begin{macrocode}
\definecolor{heading}{gray}{0.85}
%    \end{macrocode}
%
% \subsubsection{Color for hyperlink}
%    \begin{macrocode}
\ifrcls@color
    \hypersetup{colorlinks}
\else
    \hypersetup{hidelinks}
\fi
%    \end{macrocode}
%
% \subsection{Reference Settings}
%
%    \begin{macrocode}
\RequirePackage[sort&compress]{natbib}
\bibliographystyle{unsrt}
\setlength{\bibsep}{0pt}
%    \end{macrocode}
%
% \subsection{Content Variable}
%
% \begin{macro}{\name}
%    \begin{macrocode}
\def\rcls@name{}
\newcommand\name[1]{\def\rcls@name{#1}}
%    \end{macrocode}
% \end{macro}
%
% \begin{macro}{\organization}
%    \begin{macrocode}
\def\rcls@organization{}
\newcommand\organization[1]{\def\rcls@organization{#1}}
%    \end{macrocode}
% \end{macro}
%
% \begin{macro}{\address}
%    \begin{macrocode}
\def\rcls@address{}
\newcommand\address[1]{\def\rcls@address{#1}}
%    \end{macrocode}
% \end{macro}
%
% \begin{macro}{\mobile}
%    \begin{macrocode}
\def\rcls@mobile{}
\newcommand\mobile[1]{\def\rcls@mobile{#1}}
%    \end{macrocode}
% \end{macro}
%
% \begin{macro}{\mail}
%    \begin{macrocode}
\def\rcls@mail{}
\newcommand\mail[1]{\def\rcls@mail{#1}}
%    \end{macrocode}
% \end{macro}
%
% \begin{macro}{\homepage}
%    \begin{macrocode}
\def\rcls@homepage{}
\newcommand\homepage[1]{\def\rcls@homepage{#1}}
%    \end{macrocode}
% \end{macro}
%
% \begin{macro}{\resumeurl}
%    \begin{macrocode}
\def\rcls@resumeurl{}
\newcommand\resumeurl[1]{\def\rcls@resumeurl{#1}}
%    \end{macrocode}
% \end{macro}
%
% \subsection{Custom commands}
%
% \begin{macro}{\heading}
%    \begin{macrocode}
\newcommand{\heading}[1]{
    \colorbox{heading}{
        \parbox{.96\textwidth}{
            \bfseries\zihao{4}#1
        }
    } \\
}
%    \end{macrocode}
% \end{macro}
%
% \begin{macro}{\entry}
%    \begin{macrocode}
\newcommand{\entry}[3]{
    \begin{tabularx}{\textwidth}{@{\hspace{#1}}#2}
        #3
    \end{tabularx}
}
%    \end{macrocode}
% \end{macro}
%
% \subsection{Style settings}
%
% \subsubsection{Redefine maketitle}
%    \begin{macrocode}
\renewcommand{\maketitle}{
    \entry{0em}{Xr}{
        \bfseries\zihao{4}\rcls@name  & \rcls@mobile \\
        \rcls@organization            & \href{mailto:\rcls@mail}{\rcls@mail} \\
        \rcls@address                 & \url{\rcls@homepage} \\
    }
}
%    \end{macrocode}
%
% \subsubsection{Header and footer settings}
%    \begin{macrocode}
\pagestyle{fancy}
\fancyhf{}
\renewcommand{\headrulewidth}{0pt}
\renewcommand{\footrulewidth}{0pt}
\ifrcls@zh
    \CTEXoptions[today=small]
    \fancyfoot[L]{\footnotesize 最后更新:\today}
\else
    \CTEXoptions[today=old]
    \fancyfoot[L]{\footnotesize Last Modified: \today}
\fi
\fancyfoot[R]{\footnotesize \url{\rcls@resumeurl}}
%    \end{macrocode}
%
% \subsubsection{Empty refname}
%
%    \begin{macrocode}
\renewcommand{\refname}{}
%    \end{macrocode}
%
% \subsection{Customization}
%
% \subsubsection{Font settings}
%
% Cause we recommend use Xe\LaTeX{} to compile, so you can easily setting
% the font to whatever you like only if it exists on you computer.
% Just add something like following code before |\begin{document}|.
%
% \iffalse
%<*lst>
% \fi
\begin{lstlisting}
\setmainfont{Times New Roman}
\setCJKmainfont{%*宋体*)}
\end{lstlisting}
% \iffalse
%</lst>
% \fi
%
% \subsubsection{Multiple pages}
%
% Currently, \resumecls doesn't support multiple pages very well. Before
% showing the method for customization I want you relize that only one page
% for resume is enough. \resumecls use |\begin{table}| and |\end{table}| to
% enclose all contents so that the whole resume will be limited in one page.
% But if you really want to have multiple pages, you can set multiple table
% environments in your tex fhile and judge the place for new page yourself.
% the code within document environment for two pages resume will looks like:
%
% \iffalse
%<*lst>
% \fi
\begin{lstlisting}
\begin{table}
Contents for Page 1
\end{table}
\newpage
\begin{table}
Contents for Page 2
\end{table}
\end{lstlisting}
% \iffalse
%</lst>
% \fi
%
% \Finale
\endinput
