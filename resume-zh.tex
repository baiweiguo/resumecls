% !Mode:: "TeX:UTF-8"
% +-----------------------------------------------------------------------------
% | File: huxuan-zh
% | Author: huxuan
% | E-mail: i(at)huxuan.org
% | Created: 2012/12/19
% | Last modified: 2012/12/19
% | Description:
% |     Chinese Resume for huxuan in LaTeX based on resumecls
% |
% | Copyrgiht (c) 2012 by huxuan. All rights reserved.
% | License GPLv3
% +-----------------------------------------------------------------------------

\documentclass[zh,color]{resumecls}

\name{扈煊 Sean Hu}
\organization{北京大学计算机科学技术研究所}
\address{北京市海淀区中关村北大街128号,100080}
\mobile{+86 189 1160 6698}
\mail{i@huxuan.org}
\homepage{http://huxuan.org}
\resumeurl{http://huxuan.org/resume-zh.pdf}

\setmainfont{Times New Roman}
\setCJKmainfont{宋体}

\begin{document}

\begin{table}

\maketitle

%%%%%%%%%%%%%%%%%%%%%%%%%%%%%%%%%%%%%%%%%%%%%%%%%%%%%%%%%%%%%%%%%%%%%%%%%%%%%%%
\heading{教育经历}
\entry{2em}{Xrp{8em}}{
    \heiti{北京大学}          & 北京 & 2012/09 -
}
\entry{4em}{lXX}{
    硕士 & 信息科学与技术学院 & 计算机应用技术 \\
}
\entry{2em}{Xrp{8em}}{
    \heiti{北京航空航天大学}  & 北京 & 2008/09 - 2012/06
}
\entry{4em}{lXX}{
    学士 & 软件学院           & 软件工程 \\
    学士 & 数学与系统科学学院 & 数学与应用数学 \\
}

%%%%%%%%%%%%%%%%%%%%%%%%%%%%%%%%%%%%%%%%%%%%%%%%%%%%%%%%%%%%%%%%%%%%%%%%%%%%%%%
\heading{工作经历}
\entry{2em}{Xp{8em}}{
    \heiti{我在信息技术(北京)有限责任公司}       & 2011/08 - 2012/01 \\
}
\entry{4em}{X}{街旁网 \quad 研发组实习生}
\entry{6em}{X}{
    参与服务器端和底层API 的开发以及相关功能如前台的维护 \\
    主要完成子项目:同步模块的更新和重构、守护进程监控脚本 \\
}

%%%%%%%%%%%%%%%%%%%%%%%%%%%%%%%%%%%%%%%%%%%%%%%%%%%%%%%%%%%%%%%%%%%%%%%%%%%%%%%
\heading{科研经历}
\entry{2em}{Xp{8em}}{
    \heiti{北京大学计算机科学技术研究所}           & 2010/08 - \\
}
\entry{4em}{X}{网络内容保护与文档处理研究室 \quad 实习生/学生}
\entry{6em}{X}{
    版式文档的公式识别和基于MathML和\LaTeX{}的数学公式检索 \\
    参与发表论文\cite{lin2012performance,lin2012identification,
        lin2011mathematical,gao2011metadata} \\
}
\entry{2em}{Xp{8em}}{
    \heiti{北航软件学院大学生科研训练计划(SRTP)} & 2009/12 - 2010/12
}
\entry{4em}{X}{DTN网络的T连通性研究 \quad 项目立项人}
\entry{6em}{X}{
    提出DTN网络下的T连通性概念,比较不同场景下一般DTN协议的性能 \\
    参与发表论文\cite{wang2010mobile} \\
}

%%%%%%%%%%%%%%%%%%%%%%%%%%%%%%%%%%%%%%%%%%%%%%%%%%%%%%%%%%%%%%%%%%%%%%%%%%%%%%%
\heading{校园经历}
\entry{2em}{Xp{8em}}{
    北京大学计算概论课程助教                & 2012/09 -         \\
    北京航空航天大学GoogleCamp社团社长      & 2009/09 - 2011/01 \\
    北京航空航天大学开源俱乐部成员          & 2009/09 -         \\
    北航软件学院高级程序语言设计(1)课程助教 & 2009/09 - 2010/01 \\
    北航软件学院数据结构课程助教            & 2010/03 - 2010/07 \\
}

%%%%%%%%%%%%%%%%%%%%%%%%%%%%%%%%%%%%%%%%%%%%%%%%%%%%%%%%%%%%%%%%%%%%%%%%%%%%%%%
\heading{获得荣誉}
\entry{2em}{Xr}{
    北京大学研究生学业奖学金二等奖           & 2012/09 \\
    北京航空航天大学学科竞赛奖学金二等奖     & 2011/12 \\
    美国大学生数学建模比赛 (MCM 2011) 二等奖 & 2011/04 \\
    北京航空航天大学软件学院学习奖学金三等奖 & 2010/12 \\
    北京航空航天大学软件学院工作奖学金三等奖 & 2010/12 \\
}

\end{table}
\newpage
\begin{table}

%%%%%%%%%%%%%%%%%%%%%%%%%%%%%%%%%%%%%%%%%%%%%%%%%%%%%%%%%%%%%%%%%%%%%%%%%%%%%%%
\heading{专业技能}
\entry{2em}{lX}{
    精通 & Python,LaTeX,Matlab/Octave \\
    熟悉 & C/C++,Shell,JAVA,Android,PHP \\
    掌握 & 服务器的基本配置和维护(Nginx,Apache,MySQL,MongoDB) \\
    使用 & vim、Linux(Ubuntu, Arch, Gentoo) \\
    通过 & 英语四六级 \\
}

%%%%%%%%%%%%%%%%%%%%%%%%%%%%%%%%%%%%%%%%%%%%%%%%%%%%%%%%%%%%%%%%%%%%%%%%%%%%%%%
\heading{网络资料}
\entry{2em}{lX}{
    Github   & \url{http://github.com/huxuan/}         \\
    Twitter  & \url{http://twitter.com/huxuan/}        \\
    Fanfou   & \url{http://fanfou.com/huxuan/}         \\
    Douban   & \url{http://douban.com/people/huxuan/}  \\
    Google+  & \url{http://gplus.to/huxuan/}           \\
    Jiepang  & \url{http://jiepang.com/huxuan/}        \\
    Facebook & \url{http://facebook.com/huxuan.org/}   \\
    Weibo    & \url{http://weibo.com/victorhu/}        \\
}

%%%%%%%%%%%%%%%%%%%%%%%%%%%%%%%%%%%%%%%%%%%%%%%%%%%%%%%%%%%%%%%%%%%%%%%%%%%%%%%
\heading{个人喜好}
\entry{2em}{lX}{
    体育 & 健身、自行车、篮球、乒乓球、跑步      \\
    阅读 & 哲学、心理、逻辑、文学、专业相关      \\
    IT   & 开源、Linux、Google、Android          \\
    其他 & 咖啡、旅行、轻/古典音乐、摄影、思考  \\
}

%%%%%%%%%%%%%%%%%%%%%%%%%%%%%%%%%%%%%%%%%%%%%%%%%%%%%%%%%%%%%%%%%%%%%%%%%%%%%%%
\heading{附:发表论文}
\vspace{-6em}
\bibliography{huxuan}
\end{table}
\end{document}
